\documentclass[11pt, fleqn]{amsart}    
\usepackage[T1]{fontenc} 
%\usepackage[latin1]{inputenc}
%\usepackage[english,francais]{babel}
\usepackage[retainorgcmds]{IEEEtrantools}   \usepackage{graphicx}
\usepackage{amssymb}
\usepackage{pstricks}
\usepackage{pst-slpe}
\usepackage{auto-pst-pdf}
\usepackage{pst-node}
\usepackage{pst-plot}
\usepackage{cancel}
\usepackage{empheq}
\usepackage{pst-grad,multido}
\usepackage{graphicx,url,etoolbox}
\usepackage{pifont}
\usepackage{subfig}
\usepackage{siunitx}
\usepackage{mismath}
\usepackage{mathtools}
\usepackage{pstricks-add}
\usepackage{filecontents}
\usepackage{enumerate}
%\usepackage{clrscode3e}
%\usepackage{listings}
\usepackage{minted}
%\DeclarePairedDelimiter{\abs}{\lvert}{\rvert}
\usepackage[new]{old-arrows}
\usepackage{amsthm}
\DeclarePairedDelimiter\ceil{\lceil}{\rceil}
\DeclarePairedDelimiter\floor{\lfloor}{\rfloor}

\setlength{\textwidth}{\paperwidth}
\addtolength{\textwidth}{-2in}
\calclayout

\allowdisplaybreaks[1]
\theoremstyle{definition}
\newtheorem*{Lemma}{Lemma}
\theoremstyle{definition}
\newtheorem*{Theorem}{Theorem}
\theoremstyle{definition}
\newtheorem*{Definition}{Definition}
\newcommand{\iu}{{i\mkern1mu}}

\patchcmd{\section}{\scshape}{\bfseries}{}{}
\makeatletter
\renewcommand{\@secnumfont}{\bfseries}
\makeatother
\newcommand{\overbar}[1]{\mkern 1.5mu\overline{\mkern-1.5mu#1\mkern-1.5mu}\mkern 1.5mu}
%\newcommand{\Var}{\mathrm{Var}}
%\newcommand{\Cov}{\mathrm{Cov}}
\renewcommand{\Re}{\operatorname{Re}}
\renewcommand{\Im}{\operatorname{Im}}
\newcommand{\ud}{\,\mathrm{d}}
\newcommand{\p}{\,\mathrm{P}}

\usepackage{stmaryrd}
\usepackage{lipsum}
\usepackage{braket}
\usepackage{fontawesome5}
\usepackage[colorlinks=true]{hyperref}
\hypersetup{
    allcolors = [rgb]{0,0,0.5}
}
\DeclareGraphicsRule{.tif}{png}{.png}{`convert #1 `dirname #1`/`basename #1 .tif`.png}

\addtolength{\voffset}{-3cm}
\addtolength{\textheight}{4cm}
\newcommand{\eq}[1]{\begin{IEEEeqnarray*}{rCl}#1\end{IEEEeqnarray*}}
\makeatletter
\patchcmd{\@settitle}{center}{flushleft}{}{}
\patchcmd{\@settitle}{center}{flushleft}{}{}
\DeclareMathOperator{\Tr}{Tr}
%\DeclareMathOperator{\det}{det}
%\patchcmd{\@setauthors}{\centering}{\raggedright}{}{}
\patchcmd{\abstract}{3pc}{0pt}{}{} % remove indentation
\patchcmd{\section}{\centering}{}{}{}
\newcommand{\norm}[1]{\left\lVert#1\right\rVert}
\renewcommand{\abstractname}{Note} 
\makeatother

\title{%
		Concours Centrale Supelec 2023 \\
		Mathématiques 2 - MP}

\begin{document}
\maketitle
\author{Pierre-Paul TACHER}



This document is published under the \href{https://creativecommons.org/licenses/by-nc-sa/4.0/}{Creative Commons Attribution-NonCommercial-ShareAlike 4.0 International license}. \faCreativeCommons\ \faCreativeCommonsBy\ \faCreativeCommonsNc\ \faCreativeCommonsSa

\section{}\label{1}
\begin{IEEEeqnarray*}{rrCl}
&\forall t\in [0,1],\quad 1+t^2 &\leqslant & 2  \\
\Rightarrow&\forall t\in [0,1],\quad (1+t^2)^n &\leqslant & 2^n  \\
\Rightarrow&\forall t\in [0,1],\quad \frac{1}{(1+t^2)^n} &\geqslant & \frac{1}{2^n}  \\
\Rightarrow&\int_0^1 \frac{1}{(1+t^2)^n} \ud t&\geqslant & \frac{1}{2^n}  \\
\end{IEEEeqnarray*}

\section{}
La fonction
\begin{IEEEeqnarray*}{rCl}
f:\mathbb{R}^+ &\to & \mathbb{R}^{+*}  \\
t & \mapsto & \frac{1}{(1+t^2)^n}
\end{IEEEeqnarray*}
est continue par morceaux, et
\begin{IEEEeqnarray*}{rCl}
\forall t \in \mathbb{R}^{+*},\quad 0\leqslant\frac{1}{(1+t^2)^n} &\leqslant & \frac{1}{t^{2n}} \\
\end{IEEEeqnarray*}
$t\mapsto \frac{1}{t^{2n}}$ étant intégrable en $+\infty$ car $2n>1$, $f$ est intégrable et $K_n$ est bien définie.

Soit $x>0$.
\begin{IEEEeqnarray*}{rCl}
\int_0^x  \frac{1}{1+t^{2}} \ud t&= & \left[\arctan t \right]_0^x \\
&= & \arctan x \underset{x\to+\infty}{\to} \frac{\pi}{2}\\
\end{IEEEeqnarray*}
\begin{IEEEeqnarray*}{rCl}
K_1 &= &\frac{\pi}{2}  \\
\end{IEEEeqnarray*}

\section{}
On utilise l'inégalité de convexité:
\begin{IEEEeqnarray*}{rCl}
\forall t \in \mathbb{R}^{},\quad 1+t^2 &\geqslant & 2t \\
\end{IEEEeqnarray*}

Soit $x\geqslant 1$, $n\geqslant 2$.
\begin{IEEEeqnarray*}{rCl}
\int_1^x \frac{1}{(1+t^2)^n} \ud t& \leqslant & \int_1^x \frac{1}{2^nt^n} \ud t  \\
& = &\frac{1}{2^n}  \left[ -\frac{1}{n-1}\frac{1}{t^{n-1}}\right]_1^x \\
& = &\frac{1}{2^n} ( -\frac{1}{n-1}\frac{1}{x^{n-1}}+\frac{1}{n-1})\\
& \leqslant &\frac{1}{(n-1)2^n}  \\
\end{IEEEeqnarray*}
Donc:
\begin{IEEEeqnarray*}{rCl}
\int_1^{+\infty} \frac{1}{(1+t^2)^n} \ud t & =&\underset{x\to+\infty}{\lim} \int_1^x \frac{1}{(1+t^2)^n} \ud t  \\
& \leqslant &\frac{1}{(n-1)2^n} \\
\end{IEEEeqnarray*}
montre que 
\begin{IEEEeqnarray*}{rCl}
\int_1^{+\infty} \frac{1}{(1+t^2)^n} \ud t&=&\bigo(\frac{1}{n2^n})   \\
\end{IEEEeqnarray*}

\section{}
On a:
\begin{IEEEeqnarray*}{rCl}
K_n & =& \int_0^{1} \frac{1}{(1+t^2)^n} \ud t + \int_1^{+\infty} \frac{1}{(1+t^2)^n} \ud t \\
& =& I_n + O(\frac{1}{n2^n}) \\
\end{IEEEeqnarray*}
puis d'après~\ref{1}, 
\begin{IEEEeqnarray*}{rCl}
K_n & =& I_n + \underbrace{O(\frac{I_n}{n})}_{\lito{(I_n)}} \\
\end{IEEEeqnarray*}
i.e.
\begin{IEEEeqnarray*}{rCl}
K_n &\underset{n\to+\infty}{\sim} & I_n \\
\end{IEEEeqnarray*}

\section{}
Soit $x>0$, $n\in\mathbb{N}^*$. On fait une intégration par parties:
\begin{IEEEeqnarray*}{rCl}
\int_0^{x} \frac{1}{(1+t^2)^n} \ud t &= & \left[ \frac{t}{(1+t^2)^n}\right]_0^x +2n\int_0^{x} \frac{t^2}{(1+t^2)^{n+1}} \ud t \\
&= & \frac{x}{(1+x^2)^n} +2n\int_0^{x} \frac{(1+t^2)-1}{(1+t^2)^{n+1}} \ud t \\
&= & \frac{x}{(1+x^2)^n} +2n(\int_0^{x} \frac{1}{(1+t^2)^{n}} \ud t -\int_0^{x} \frac{1}{(1+t^2)^{n+1}} \ud t )\\
\end{IEEEeqnarray*}
En passant à la limite quand $x\to+\infty$, on obtient:
\begin{IEEEeqnarray*}{rCl}
K_n &= &2n(K_n-K_{n+1})  \\
\end{IEEEeqnarray*}

\section{}\label{6}
On réécrit la récurrence précédente
\begin{IEEEeqnarray*}{rCl}
 K_{n+1}&= &\frac{2n-1}{2n}K_n  \\
\end{IEEEeqnarray*}
D'où il vient facilement que 
\begin{IEEEeqnarray*}{rCl}
K_n &= & \frac{(2n-3)(2n-5)\dots 3\times 1}{(2n-2)(2n-4)\dots 4\times2}K_1 \\
&= & \frac{(2n-3)(2n-5)\dots 3\times 1}{2^{n-1}(n-1)!}\frac{\pi}{2} \\
&= & \frac{(2n-2)!}{2^{2(n-1)}(n-1)!^2}\frac{\pi}{2} \\
\end{IEEEeqnarray*}

On applique la formule de Stirling
\begin{IEEEeqnarray*}{rCl}
(2n-2)! & \underset{n\to+\infty}{\sim}& (\frac{2(n-1)}{e})^{2(n-1)}\sqrt{4\pi(n-1)} \\
\end{IEEEeqnarray*}
et
\begin{IEEEeqnarray*}{rCl}
(n-1)!^2 &\underset{n\to+\infty}{\sim} & (\frac{n-1}{e})^{2(n-1)}2\pi(n-1)  \\
\end{IEEEeqnarray*}
Après simplification on obtient
\begin{IEEEeqnarray*}{rCl}
K_n & \underset{n\to+\infty}{\sim}& \frac{\sqrt{\pi}}{2\sqrt{n}} \\
\end{IEEEeqnarray*}

\section{}
On effectue le changement de variable $u=\sqrt{n}t=\phi(t)$. On a $\phi'(t)=\sqrt{n}$,
\begin{IEEEeqnarray*}{rrCl}
&I_n &= &\int_0^{\sqrt{n}} \frac{1}{(1+\frac{u^2}{n})^n}\frac{1}{\sqrt{n}}\ud u  \\
\Rightarrow&\sqrt{n}I_n &= &\int_0^{\sqrt{n}} \frac{1}{(1+\frac{u^2}{n})^n}\ud u  \\
\end{IEEEeqnarray*}

\section{}
On pose:
\begin{IEEEeqnarray*}{rCl}
f_n:\mathbb{R}^+ &\to & \mathbb{R}^{+*}  \\
u &\mapsto & \left\{ \,
  \begin{IEEEeqnarraybox}[][c]{l?s}
    \IEEEstrut
    \frac{1}{(1+\frac{u^2}{n})^n}& si $u\in[0,\sqrt{n}]$ \\
    0 & si $u\in]\sqrt{n},+\infty[$
    \IEEEstrut
  \end{IEEEeqnarraybox}
\right.
\end{IEEEeqnarray*}
Les $f_n$ sont continues par morceaux. La fonction $x\mapsto (1+x)^n$ étant convexe sur $\mathbb{R}^+$, on a
\begin{IEEEeqnarray*}{rCl}
\forall u\in \mathbb{R},\quad (1+\frac{u^2}{n})^{n}& \geqslant & 1+n \frac{u^2}{n}\\
& = & 1+u^2
\end{IEEEeqnarray*}
donc en posant $\phi(u)=\frac{1}{1+u^2}$,
\begin{enumerate}[i]
  \item $\forall n\in \mathbb{N}^*,\quad \forall u\in \mathbb{R}^+,\quad \abs{f_n(u)}\leqslant \phi(u)$, intégrable en $+\infty$.
  \item la suite de fonction $(f_n)$ converge simplement sur $\mathbb{R}^+$ vers la fonction $u\mapsto e^{-u^2}$.
\end{enumerate}
Le théorème de convergence dominée nous permet d'affirmer que $\sqrt{n}I_n$ converge et que
\begin{IEEEeqnarray*}{rCl}
\lim_{n\to+\infty} \sqrt{n}I_n &= & \int_0^{+\infty} e^{-u^2}\ud u \\
\end{IEEEeqnarray*}

\section{}
D'après la question~\ref{6},
\begin{IEEEeqnarray*}{rCl}
\sqrt{n}I_n &\underset{n\to+\infty}{\sim} & \frac{\sqrt{\pi}}{2} \\
\end{IEEEeqnarray*}
donc
\begin{IEEEeqnarray*}{rCl}
  \int_0^{+\infty} e^{-u^2}\ud u& =& \frac{\sqrt{\pi}}{2} \\
\end{IEEEeqnarray*}

Le changement de variable $\phi(u)=\sqrt{2}u$ nous donne
\begin{IEEEeqnarray*}{rCl}
  \int_0^{+\infty} e^{-\frac{u^2}{2}}\ud u& =& \sqrt{\frac{\pi}{2}} \\
\end{IEEEeqnarray*}
puis par parité,
\begin{IEEEeqnarray*}{rCl}
 \int_{-\infty}^{+\infty} e^{-\frac{u^2}{2}}\ud u& =& 2\int_0^{+\infty} e^{-\frac{u^2}{2}}\ud u \\
 & =& \sqrt{2\pi}\\
\end{IEEEeqnarray*}
\iffalse
\newpage
{\small
\begin{thebibliography}{99}
\bibitem{gourdon} Gourdon, Xavier: 
\emph{Algèbre, 2è édition}, 
Ellipses (2009)
\end{thebibliography}
}
\fi
\end{document}

